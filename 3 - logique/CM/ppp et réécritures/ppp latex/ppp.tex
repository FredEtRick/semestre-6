\documentclass[11pt]{article}

\usepackage[utf8]{inputenc}
\usepackage[T1]{fontenc}
\usepackage[frenchb]{babel}

\usepackage[a4paper]{geometry}
% si besoin de changer les marges :
%\usepackage[a4paper,left=2cm,bottom=2cm]{geometry}

\usepackage{amsmath,amsfonts,amssymb}
%\usepackage[np]{numprint}
\usepackage{xspace}
\usepackage{fancyhdr} % suggestion pour les en-tête
\usepackage{color}
\pagestyle{fancy}

\usepackage{enumitem}
\newlist{longenum}{enumerate}{15}
\setlist[longenum,1]{label=(\Roman*), font=\color{blue}}
\setlist[longenum,2]{label=(\Alph*), font=\color{blue}}
\setlist[longenum,3]{label=(\arabic*), font=\color{blue}}
\setlist[longenum,4]{label=(\roman*), font=\color{blue}}
\setlist[longenum,5]{label=(\alph*), font=\color{blue}}
\setlist[longenum,6]{label=(\Roman*), font=\color{blue}}
\setlist[longenum,7]{label=(\Alph*), font=\color{blue}}
\setlist[longenum,8]{label=(\arabic*), font=\color{blue}}
\setlist[longenum,9]{label=(\roman*), font=\color{blue}}
\setlist[longenum,10]{label=(\alph*), font=\color{blue}}
\setlist[longenum,11]{label=(\Roman*), font=\color{blue}}
\setlist[longenum,12]{label=(\Alph*), font=\color{blue}}
\setlist[longenum,13]{label=(\arabic*), font=\color{blue}}
\setlist[longenum,14]{label=(\roman*), font=\color{blue}}
\setlist[longenum,15]{label=(\alph*), font=\color{blue}}

\begin{document}

	\begin{longenum}
		\item \color{red} (début cours 1) \color{black} logique des propositions \(\rightarrow\) logique des prédicats du premier ordre
		\begin{longenum}
			\item intro
			\begin{longenum}
				\item la logique
				\begin{longenum}
					\item science du raisonnement
					\item but
					\begin{longenum}
						\item évaluer la validité ou l'invalidité d'un raisonnement
						\item déterminer si des hypothèses peuvent amener à une conclusion
						\item à partir d'un raisonnement, dire s'il est ou non correct
					\end{longenum}
					\item exemple
					\begin{longenum}
						\item A = "albert est coupable"
						\item B = "baptiste est coupable"
						\item C = "albert avait un couteau"
						\item D = "celui qui avait un couteau a tué tout le monde"
						\item de \(C\land D\) je déduis \(\neg A\)
						\item le rôle de la logique est de dire que ce raisonnement n'est pas valide
					\end{longenum}
				\end{longenum}
				\item "niveaux" de logique
				\begin{longenum}
					\item 
					\item les années passées : logique des propositions
					\item cette année : logique des prédicats du premier ordre
				\end{longenum}
				
			\end{longenum}
		\end{longenum}
	\end{longenum}

\end{document}